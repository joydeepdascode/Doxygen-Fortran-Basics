\label{md_code_subpage2_subpage2}%
\Hypertarget{md_code_subpage2_subpage2}%
 {\bfseries{Using Doxygen to create an Introduction to the source code(\+A Markdown File)\+:}}

\DoxyHorRuler{0}
 {\itshape Initially, when starting to create documentation of the code, one feels to give a short introduction of the problem in hand and related stuffs. Also, a need to organize the pages in the Doxygen project, by creating a table of contents say. This is the main content to be discussed in this topic.}

\DoxyHorRuler{0}


Follow the below steps,
\begin{DoxyEnumerate}
\item {\bfseries{To create a mainpage}}\+: Create a mainpage file in the Doxygen input directory. And inside teh file enter, \begin{DoxyVerb}@mainpage <title_of_the_Mainpage>

...Give the introduction to the problem in the project.. \end{DoxyVerb}

\item {\bfseries{To create a page}}\+: Create a page1.\+md file in the Doxygen input directory. And inside the file enter, \begin{DoxyVerb}@page <page_identifier> <title_of_the_page>

...Write a description of this page project.. \end{DoxyVerb}

\end{DoxyEnumerate}
\begin{DoxyEnumerate}
\item {\bfseries{To create a page with Table of Contents}}\+: Create a \mbox{\hyperlink{subpage1_8md}{subpage1.\+md}} file in the Doxygen input directory. Inside the file add the {\ttfamily @}tableofcontents command to generate table of contents bar in the webpage.
\item {\bfseries{To create a section in page}}\+: Say, you want to add a section in either the mainpage or subpage, to do that,
\end{DoxyEnumerate}
\begin{DoxyEnumerate}
\item {\bfseries{In the mainpage\+:}} Type the following \begin{DoxyVerb}@section <section_identifier> <Section_name>
\end{DoxyVerb}

\item {\bfseries{In a subpage\+:}} Say, you create a subpage \char`\"{}page1.\+md\char`\"{}, to create a section type 
\end{DoxyEnumerate}